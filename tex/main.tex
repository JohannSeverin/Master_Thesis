\input{tex/tufte_config/tufte_preamble.tex}

\setcounter{secnumdepth}{2}
\setcounter{tocdepth}{1}

\let\cleardoublepage\clearpage


\usepackage[utf8]{inputenc}
\usepackage{subfiles}
\usepackage{nameref}
\usepackage{lastpage}
\usepackage{physics}
\usepackage{amsmath}


% Command
\newcommand{\pfrac}[2]{\frac{\partial #1}{\partial #2}}
\newcommand{\pfracsquared}[2]{\frac{\partial^2 #1}{\partial #2^2}}
\newcommand{\unitary}[0]{\mathcal{U}}
\newcommand{\fracsqrttwo}[0]{\frac{1}{\sqrt{2}}}



% Colors 
% https://coolors.co/5f0f40-873b5e-af677c-ffbfb7-ffd447-48a9a6-243010
\definecolor{color1}{HTML}{5F0F40}
\definecolor{color2}{HTML}{873B5E}
\definecolor{color3}{HTML}{AF677C}
\definecolor{color4}{HTML}{ffbfb7}
\definecolor{color5}{HTML}{ffd447}
\definecolor{color6}{HTML}{48a9a6}
\definecolor{color7}{HTML}{243010}

% Colors of stuff
\hypersetup{
    linkcolor = color6
}

%\renewcommand{\chaptermark}[1]{\markleft{\thechapter}}

% Header on pages
\fancypagestyle{fancy}{
  \fancyhf{} 
  \fancyhead[L]{\leftmark}
  \fancyhead[R]{\thepage}
}

% Header on Chapter Opening
\fancypagestyle{plain}{
  \fancyhf{} 
   \fancyfoot[R]{\thepage}
}

% chapter format
\titleformat{\chapter}%
  {\huge\bf\color{color1} }% format applied to label+text
  {\llap{\colorbox{color1}{\parbox{1.5cm}{\hfill\huge\bf\color{white}\thechapter}}\hspace{0.2cm}}}% label
  {0 cm}% horizontal separation between label and title body
  {}% before the title body
  []% after the title body

\titleformat{\section}%
  {\LARGE\bf\color{color2} }% format applied to label+text
  {\llap{\colorbox{color2}{\parbox{1.0cm}{\hfill\LARGE\bf\color{white}\thesection}}\hspace{0.2cm}}}% label
  {0 cm}% horizontal separation between label and title body
  {}% before the title body
  []% after the title body

\titleformat{\subsection}%
  {\bf\color{color3}}% format applied to label+text
  {\llap{{\hfill\bf\color{color3}\thesubsection}\hspace{0.2cm}}}% label
  {0 cm}% horizontal separation between label and title body
  {}% before the title body
  []% after the title body



\title{Superconducting \\ \\Qubit Readout}
\author{Johann Bock Severin}
\subtitle{In theory, experiment and \\simulation}
\publisher{Supervised by:\hfill Morten Kjaergaard\\ \phantom{hey}\hfill Jacob Hastrup  \\ \\ Handed in:\hfill \today}
\date{\today}

\begin{document}
% \subfile{TODO.tex}
% Make title page
% \BgThispage
\pagenumbering{roman}
\maketitlepage

% To do notes
\chapter*{\centering Abstract}
For superconducting qubits to be a viable platform for large scale quantum information processing, high-fidelity readout is required. This thesis will look into the underlying physics describing the system and time evolution in an initialization and readout sequence in order to study, how different physical parameters contribute to the State Preparation and Measurements (SPAM) errors. By calibrating a single superconducting qubit, a simulation model is built using the stochastic master equation to simulate the dispersive approximation of a qubit-resonator system. The model is capable of producing realistic plots of IQ measurements which have similar distributions and SPAM fidelity as measured in the laboratory. The model is used to estimate the contribution to the infidelity from three factors: non-zero temperature, energy decay during measurement and inefficient measurement. We conclude that non-zero temperature is the biggest contributor in the analyzed system. The model is further used to simulate the system with marginal improvements . This serves as a basis for discussing improvements of superconducting qubit readout.


% To implement large scale quantum information protocols on superconducting qubits, high-fidelity readout is required. 

% Since state initialization and measurement errors are inseparable in quantum algorithms, this thesis goes into the physics in the readout process. 

% To fully map the system a list of calibrations are worked out. 

% By introducing the dispersive approximation for a qubit resonator system and simulating it using the stochastic master equation, this thesis uses the calibrated parameters to make realistic IQ plots which resemble that of the experimental setup. 

% The thesis uses the models to estimate the contributions to state preparation and measurements errors from three sources: non-zero temperature, energy decay during measurement and inefficient measurement.  

% Finally, a use case is presented, where marginal improvements are mapped to expected SPAM fidelities. This is used to spark a discussion of how one can improve the readout of superconducting qubits. zz 


% \vspace{2 cm}



% \todo{This is a chatgpt abstract, rewrite it so it is not cheating.}
% \thispagestyle{empty}
% This study delves into the theory, calibration, and simulation of readout in a superconducting system. The theoretical framework encompasses three key stages: unitary evolution, elucidating the coupling between qubit and resonator for qubit measurement; interaction with the environment, addressing challenges arising from coherence loss and energy exchange; and stochastic evolution, depicting single trajectory unraveling. These theoretical underpinnings provide essential tools for dynamic simulation, contingent upon a set of determinable parameters for a realistic model.

% Calibration efforts yield a comprehensive list of parameter estimates for integration into the simulation model. Comparative analysis against laboratory readout sequences reveals similar state initialization and measurement fidelity, although certain features of the distribution hint that further scrutiny is required, particularly regarding X-gate fidelity and proper inclusion of the second excited state.

% Optimal parameter settings enable an assessment of three distinct sources contributing to SPAM (State Preparation and Measurement) infidelity. Results indicate that, for our system employing a 600 ns readout pulse, the primary source of infidelity stems from high temperature, followed by low efficiency and diminished coherence time.

% Furthermore, the model serves as a platform for estimating SPAM-fidelity enhancements through marginal improvements in efficiency, temperature, or coherence time. These estimates serve as a foundation for discussing strategies for initialization and readout sequences, offering insights into mitigating the impact of low efficiency, elevated temperature, or diminished coherence, provided that other parameters are optimized.

\chapter*{Acknowledgements}
This thesis would not have been, if were it not for my two truly fantastic supervisors, Jacob Hastrup and Morten Kjaergaard. I deeply appreciate the advice, cooperation and fruitful discussions, we have had along the way. 

I also want to extend a special thanks to the competent and kindhearted people at SquidLab, who have provided me with technical support, theoretical discussion and a perhaps unhealthy amount of coffee breaks. A particular thanks should go to Malthe Nielsen, who has been my guide in the laboratory and have been building the software to support whatever experiment I wanted to do.

In the would outside the laboratory, I want to thank friends, family and my girlfriend who have been encouraging, supporting and overbearing while I immersed myself in the field of Superconducting Qubits.

With the submission of this thesis, I now conclude five amazing years learning physics at the Niels Bohr Institute. I am grateful for all the people I met along the way, I could never have hoped for such amazing people to work, study, laugh, live and be myself among. 



% Setup TOC and proper page numbers
\newpage
\tableofcontents

%%% Include files
% Introduction
\pagestyle{fancy}
\newpage
\pagenumbering{arabic}
\setcounter{page}{1}

\subfile{sections/1_introduction.tex}

\subfile{sections/2_cQED.tex}

\subfile{sections/3_computations_and_readout.tex}

\subfile{sections/4_quantum_dynamics.tex}

\subfile{sections/5_measurements.tex}

\subfile{sections/6_reaodut_strategies.tex}

\subfile{sections/7_calibration_from_IQ.tex}

\subfile{sections/8_building_model.tex}

\subfile{sections/9_budget.tex}

\subfile{sections/10_conclusion.tex}


\begin{fullwidth}
\newpage
\addcontentsline{toc}{chapter}{Bibliography}
\bibliographystyle{unsrtnat}
\bibliography{tex/Master_Thesis.bib}
% \bibliography{\jobname}
% \bibliographystyle{plain}
\end{fullwidth}

\appendix

\subfile{tex/appendix/A1_code_documentation.tex}

\subfile{tex/appendix/A2_multiqubit.tex}

\subfile{tex/appendix/A3_fit_params.tex}

\subfile{tex/appendix/A4_IQ_distributions.tex}

% \subfile{tex/appendix/A3_cont_model_prior.tex}


\end{document}