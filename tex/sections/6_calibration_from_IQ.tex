\chapter{Calibration Methods}
This sections for now serves to be a "dump", where I can put the methods/results that I made myself. 


\section{Qubit Calibration}
Before using a qubit, it will be necessary to characterize it to the best of our ability, in this section, we will go over the standard methods for characterizing the necessary qubit and resonator properties. 

\subsection{Spectroscopy}
To find 

\begin{itemize}
    \item Qubit 
\end{itemize}

\subsection{Rabi}

Use this to find:
\begin{itemize}
    \item Proper pulses
    \item T1
\end{itemize}


\subsection{Decay Calibration}
How do we determine the calibration of T1. 

We get some measure from Rabi oscillations. But we follow this algorithm:

\begin{enumerate}
    \item Perform an x-gate
    \item Wait a time: $t$ 
    \item measure 
    \item repeat 1-3 $n$ times
    \item repeat 1-4 with increased $t$
\end{enumerate}

Fitting an exponential decay to the occupation in the excited state will give a measure 

\subsection{Dephasing Calibration}
There are two ways of calibrating the dephasing time. \textbf{Refer to the section in Lindblad}.

\paragraph{Ramsey Experiment}
This is a representative time of the qubit.

\paragraph{Echo Experiememt}
This is the limit if we allow activly chanigning the qubit to echo out the phases. We can thus eliminate low frequency noise.


\section{Resonator Calibration}
\subsection{Spectroscopy}
\begin{itemize}
    \item Single dip
    \item Double dip - dispersive shift
\end{itemize}

\subsection{Photon Counting}
Considering the dispersive hamiltonian of the qubit-resonator-system, Eq. \ref{eq:two_level_qubit_dispersive}, we have above primarily considered the shift of resonator frequency depending on the qubit state, but just moving the parenthesis to represent it in the form:
\begin{equation}
    H = \Tilde{\omega}_r a^\dagger a  + \left(\frac12 \Tilde{\omega}_{01} + \chi a^\dagger a\right)  \sigma_z
\end{equation}
making it apparent that while the qubit state moves the resonator, the opposite is also true. Since the qubit frequency moves with $\chi$ for each photon in the resonator, the photon number at a given amplitude can be calibrated.  

The protocol is carried out in the following way:
\begin{itemize}
    \item Drive the resonator at a given amplitude to fill it with photons.
    \item When the resonator is in its steady state apply an X-gate with a given frequency.
    \item Perform a regular readout
\end{itemize}
By sweeping over amplitude and X-gate frequency, one can find the resonance frequency at every amplitude. Since the resonance frequency is related to the qubit \\

\begin{marginfigure}[- 5 cm]
    \centering
    \includegraphics[width = 1.3 \textwidth]{tex/fig_for_text/section_6/photon_number_calibration.png}
    \caption{Simulation of photon number calibration protocol.  }
    \label{fig:photon_number_calibration}
\end{marginfigure}

\noindent
\textbf{We need the following Data}:
\begin{itemize}
    \item First calibrate the double dip and find the dispersive shift from above
    \item For the $\ket{0}$, we should drive the resonator at the resonator frequency and measure the $|I + iQ|$
    \item For the $\ket{1}$, we fill the resonator with photons and apply an x-gate with a given frequency. Scan over this frequency to find the resonance frequency as a function of amplitude $\to |I + iQ|$ measured above
    \item We should repeat this experiment where we apply another readout frequency to the most optimal (probably in between the two peaks). 
\end{itemize}

\subsection{Readout Efficiency}
Can we do this? Would be amazing for my knowledge about the resonator and to calibrate IQ plane.

I'm pretty sure we can do this by comparing the noise the standard deviation of the signal with the mean. With perfect efficiency the ratio should be given. When comparing our relation, we can probably come with a good bet on the difference.

\section{Calibrations in IQ-plane - Single Shot}
In this section the idea is, if one can calibrate some of the fundamental parts about our qubit by fitting distributions in the IQ-plane for readouts at different duration. We wait for the qubit to come to rest. Ideally in $\ket{0}$ but as the fridge is at non-zero temperature, we will also find a part of the population in $\ket{1}$. The experiment goes as follows.
\begin{itemize}
    \item Wait for equilibrium.
    \item For 50\% of the experiments apply an $X$-gate. Label the flipped qubits for $\ket{1}$ and the rest $\ket{0}$
    \item Measure for a set duration. This is done by applying a drive to the resonator at the dressed resonator frequency when the qubit is in state $\ket{0}$: $\omega_{drive} = \omega_r - \chi$
\end{itemize}
What we expect to see is that the qubit states of $\ket{1}$ will be pushed away. \footnote{This is beacuse we measure reflection/trasmission. We're free to perform displacement and rotations in the IQ plane since only magnitudes and phase \textbf{differences} have physical meaning.} 

\subsection{Theoretical Assumptions}
We now assume that the driving duration is so short that we will not see the qubits rotating back. Should not happen when driving in resonance with the one dressed frequency though. We will further assume, that a steady state is reached instantaneously. The state of the resonator in the IQ planer will then be constant unless the qubit decays/excites. 

We further assume that the characteristic decay times are much longer that the readout durations. This means: we can neglect the possibilities of decaying twice. For example $\ket{1} \to \ket{0} \to \ket{1}$ is prohibited in this simple model and we can assume that the decay process is linear.

These assumptions are a bit of a stretch but allow us a first analysis, where we only have to worry about doing a few linear fits.

\subsection{The Data}
The signal from the IQ plot is recorded in the digitizer and integrated up over the period. This experiment is done with a few different durations of measurement at $0.5, 1, 2, 5$ and $10 \mu s$. It is recorded wether an X-gate was applied before the measurement or not. An example of the data after $0.5 \mu s$ is given in Figure \ref{fig:raw_data_initial_IQ}.

\begin{figure}
    \centering
    \includegraphics{Figs/Results/IQ_plane_initial/IQ_raw_at_0.5.png}
    \caption{The distributions from a heterodyne measurement of $0.5 \mu s$. The two distributions are from applying an X-gate or not, such that to the best of our knowledge the system is in state $\ket{0}$ or $\ket{1}$.}
    \label{fig:raw_data_initial_IQ}
\end{figure}


\subsection{Compounded Gauss with Uniform Mean}
For the T1 calibration, we need a distribution for the decayed state lying along the line between the two other distributions. Since every point comes with a Gaussian distribution around some mean, we will model this as a gaussian with a traveling mean.

As a first approach and to first order, where the growth in population is constant during the whole travel distance, we can take the mean to be uniformly taken from between the two Gaussians. We are given the mean for the two distributions: $\boldsymbol{\mu_1} = (\mu_{x1}, \mu_{y1})$ and $\boldsymbol{\mu_2} = (\mu_{x2}, \mu_{y2})$. To create a uniformly distribution between them, we will rotate and translate to place $\boldsymbol{\mu_1}$ in the origin and have $\boldsymbol{\mu_2}$ extend outwards along the x-direction. We perform the translation:
\begin{align*}
    x' = (x - \mu_{1x}) \hspace{1 cm} y' = y - \mu_{1y}
\end{align*}
and the rotation:
\begin{align*}
    \Tilde{x} = \cos\theta \; x' - \sin\theta \; y' \hspace{1cm} \Tilde{y} =  \sin\theta \; x' + \cos\theta \; y' 
\end{align*}
Where the angle is decided by:
\begin{equation}
    \tan \theta = - \frac{\mu_{y2}'}{\mu_{x2}'} 
\end{equation}
such that  $\boldsymbol{\mu_{2}}$ is on the x axis with $\Tilde{x} = |\boldsymbol{\mu_{2}} - \boldsymbol{\mu_{1}}|$.

In this new coordinate system, we have $\Tilde{\mu_{x2}}$ uniformly distributed from origin to $\mu_2$. If we now start with a 2d Gaussian in the new coordinates with $\sigma_x = \sigma_y = \sigma$, it will take the form:
\begin{equation}
    \frac{1}{2 \pi \sigma} \exp \left(- \frac{(\Tilde{x} - \Tilde{\mu_x})^2 + \Tilde{y^2}}{2 \; \sigma^2} \right)
\end{equation}
where the $\mu_y = 0$ in this frame. Now we have to take $\mu_x$ uniformly from the interval between $0$ and $r = |\boldsymbol{\mu_{2}} - \boldsymbol{\mu_{1}}|$. We get the following expression:

\begin{align}
    &\frac{1}{r} \int_0^r d\Tilde{\mu_x} \frac{1}{2 \pi \sigma^2} \exp \left(- \frac{(\Tilde{x} - \Tilde{\mu_x})^2 + \Tilde{y^2}}{2 \; \sigma^2} \right) \\
    = &\frac{1}{r}\frac{1}{\sqrt{2 \pi} \sigma} \Big(\erf((r + \Tilde{x}) / \sigma) - \erf(\Tilde{x} / \sigma)\Big) \exp(-\frac{\Tilde{y}^2}{2 \sigma^2})
\end{align}
Where the integration of $\Tilde{\mu_x}$ gives the error function, defined by: 
\begin{equation}
    \erf(x/\sigma) = \frac{1}{\sqrt{2 \pi }\sigma} \int_0^x dt e^{-t^2/2\sigma^2}
\end{equation}
This distribution can now be transferred back to the original coordinates. 

\subsection{Model}
To fit this data, we will split it in three parts:
\begin{enumerate}
    \item The main Gaussian distributions. This will be two-dimensional gaussian distribution with covariance $COV = \sigma^2 \cdot \identity$. Such that it has an even distribution in both the $I$ and $Q$ direction.
    \item The initilization error, which is modelled as a gaussian distribution with same covariance as the main one, but with a different mean. This mean is guessed (but for now not forced) to be in the center of the opposite distributions main Gaussian.
    \item The decay distribution. In the limits we opposed on the system, we can model the decays as a compound distribution of a gaussian with a uniform distribution of means between the main and the wrong-initilization-Gaussian. This model is desribed in the prior subsection. The standard deviation will still be the same as in the two above.  
\end{enumerate}
\noindent
Collecting these three parts, we have a distribution function with 7 parameters:

\begin{itemize}
    \item $\boldsymbol{\mu}_{\text{main}}$, $\boldsymbol{\mu}_{\text{wrong}}$ which are the two two-dimensional describing the center of the two gaussian distributions.
    \item $\sigma$ which is the standard deviation of all distributions.
    \item $f_{\text{wrong}},\; f_{\text{decay}}$ which are the fractions of the distribution in the wrong initialized state or in the decayed part. The fraction of the main gaussian is of course given as $f_{\text{main}} = 1 - f_{\text{wrong}} -f_{\text{decay}}$.
\end{itemize}

\noindent
And the total distribution is given as:

\begin{align}
    \PDF(I, Q) &= f_{\text{main}} \cdot \MVG\left[\boldsymbol{\mu}_{\text{main}}, \sigma^2\right](I, Q) \label{eq:pdf_of_iq_plane_distribution} \\
               &+ f_{\text{wrong}} \cdot \MVG\left[\boldsymbol{\mu}_{\text{wrong}}, \sigma^2\right](I, Q) \nonumber \\
               &+ f_{\text{decay}} \cdot \MVG\left[\Uniform\left[\boldsymbol{\mu}_{\text{main}}, \boldsymbol{\mu}_{\text{wrong}}\right], \sigma^2 \right](I, Q) \nonumber
\end{align}
where $\MVG\left[\boldsymbol{\mu}, \boldsymbol{\Sigma} \right]$ is a multivariate gaussian with mean $\boldsymbol{\mu}$ and covariance matrix $\boldsymbol{\Sigma}$, and $\Uniform{\boldsymbol{x, y}}$ is a uniform distribution of the points on the line between $\boldsymbol{x}$ and $\boldsymbol{y}$.

We now fit the data at $t = 0.5 \mu s$ which is seen from Figure \ref{fig:raw_data_initial_IQ} by using the Minuit Migrad minimizer to minimize the unbinned negative log likelihood function $\nLLH = \sum_i \log \PDF(I_i, Q_i)$, where the sum over $i$ refers to summing all data points. The results can be seen in Figure \ref{fig:fit_0_5_IQ_plane}.

\begin{figure*}
    \centering
    \includegraphics{Figs/Results/IQ_plane_initial/fit_0_5.pdf}
    \caption{The data from a $0.5 \mu s$ measurement fitted with the probability density distribution given in Equation \ref{eq:pdf_of_iq_plane_distribution}. The parameters from the fractions of distributions are given. The "wrong" and "decay" fractions are from the fit, and the correct fractions are error propagated from the two others. \textbf{While the other parameters could probably be found in the appendix.}}
    \label{fig:fit_0_5_IQ_plane}
\end{figure*}



\subsection{Evolution over Time}
Repeating the same fit procedure over the time for $t = 0.5, 1, 2, 5$ and $10 \; \mu s$, we can plot the fraction in each distribution as a function of time. This is done in Figure \ref{fig:evolution_of_fractions_in_fit}. Linear fits are also performed on the fitted fractional parameters and can be seen in \ref{tab:experimental_fit_parameters_IQ_plane}. The last $10 \; \mu s$ point is removed since it violates to many of our rough assumptions made earlier, and would not give us meaningful fit parameters.

\begin{figure}
    \centering
    \includegraphics{Figs/Results/IQ_plane_initial/fraction_evolution.pdf}
    \caption[][2 cm]{Evolutioon the fitted parameters for $f_{\text{wrong}}$ and $f_{\text{decay}}$ for the two states $\ket{0}$ and $\ket{1}$ respectively. The linear fit paramters can be found in table \ref{tab:experimental_fit_parameters_IQ_plane}.}
    \label{fig:evolution_of_fractions_in_fit}
\end{figure}


\begin{table}
\centering
\begin{tabular}{l|c|c|c|c}
 & Initial Intercept & Initial Slope & $\chi^2$ & p-value \\
\hline
Ground Decay & $8.3 \pm 0.3 \%$ & $-0.5 \pm 0.1 \%$ & 3.3 & 0.353 \\
Ground Decay & $1.4 \pm 0.4 \%$ & $1.4 \pm 0.1 \%$ & 6.2 & 0.103 \\
Excited Decay & $9.2 \pm 0.5 \%$ & $0.4 \pm 0.2 \%$ & 0.9 & 0.634 \\
Excited Decay & $0.8 \pm 0.8 \%$ & $8.7 \pm 0.3 \%$ & 2.2 & 0.326 \\
\end{tabular}
\caption{Parameters and uncertainty from the linear fits of the fractions. }
\label{tab:experimental_fit_parameters_IQ_plane}
\end{table}


\subsection{Physical Parameters}
We use the above approach to do some calibration. The following parameters are possible to find:
\begin{itemize}
\item $T_{1\uparrow}$ - decay time
\item $T_{1\downarrow}$ - excitation time 
\item The equilibrium distribution $\ket{0}$
\item The x-gate fidelity 
\item The temperature should be found from both comparing the rate of excitation/decay and from the equilibrium distribution.
\end{itemize}

The actual distributions for these will be that the distribution will follow $(1 - e^{-t/T_1})$, or to first order $(t/T_1)$, which is what we have used in this fits. This is of course an approximation to low order in $t/T_1$. We correlate the slope to $T_1$ by $1 / \Gamma$, where $\Gamma$ is the slope/decay rate of the fit.
We find:
\begin{itemize}
    \item Excited state decay time: T1-down: 11.45 +- 0.38 us
    \item Ground  state decay time: T1-up: 73.82 +- 0.10 us
    \item Gamma: 	 0.101 +/- 0.003 / us
    \item T1: 	 9.915 +/- 0.297 us
    \item X gate infidelty: 	 1.0 +- 0.6 %
\end{itemize}

\paragraph{Temperature}
We now have two ways to find the temperature. The first is to use the equilibrium distribution and the second is to use the decay rate.

In equilibrium, the relation between the $p_1/p_0 = \tanh(-\beta \hbar \omega / 2)$, where $\omega$ is the energy difference between the two states.

We have the same relation between the two decay rates: $\Gamma_{1\uparrow} / \Gamma_{1\downarrow} = \exp(-\beta \hbar \omega)$

Temperature: 	 127.9 +/- 2.9 mK

Temperature: 	 153.6 +/- 5.4 mK

\section{Calibration in Continuous Time}

\subsection{Transition Rates}
If we were to consider the dynamics over a longer time-interval, we would expect how to see the qubit change from $\ket{1} \leftrightarrow \ket{0}$ multiple times. If we were to consider, how often these transitions occur, we can determine the decay rate, and excitation rate allowing us to again extract $T_1$ and temperature $\tau$.

\textbf{Do analysis with data}



\subsection{In-measurement T1 }
Can we compare the decay-rate within a measurement to the decay outside. There is multiple reasons, this might change:
\begin{itemize}
    \item Resonator-Qubit interaction: the resonator and qubit are coupled, so when we are filling the resonator, we allow multiple form of decay/excitations through the qubit. This will lower the expect living time.
    \item Quantum Zeno Effect: Since the dynamics of the wave-function is linear but the stochastic nature of measuring is quadratic, a continuous measurement should somewhat prevent a transition. This will extent the expected lifetime of the qubit in both states with a factor related to the measurement efficiency $\eta$
\end{itemize}

\textbf{Do the experiment} where a continous measurement is made from 0 to 10 mikrosecond, followed up with another where a waiting windows is placed just before:
\begin{enumerate}
    \item Place the qubit in 0 or 1
    \item wait a number of mikroseconds
    \item do a contiouns measurement
\end{enumerate}
The ideas from before can now be used to fit the in-measurement decay rates, while we can use the points across measurements to control a more common decay rate which happens outside of measurements.