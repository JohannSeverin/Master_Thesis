\chapter{Comparison of Experiment and Simulation}
In the previous chapter, we have covered, how one calibrates and extracts the parameters necessary to model a qubit-resonator system under readout. We are now in a position to model it, however, with both many different levels of complexity and different available approximations at hand, we have a trade-off between run time of our simulations and the realism of them. 

\section{Different Simulation Approaches}
Thoughout the first few chapter, we covered different ways of representing and numerically integrating a quantum system. In this chapter, we will summarize the few methods.

\begin{itemize}
    \item \textbf{Unitary}
    \item \textbf{Lindblad Equation}
    \item \textbf{Monte Carlo}
    \item \textbf{Stochastic Master Equation}
\end{itemize}


\begin{margintable}
    \caption{Caption}
    \centering
    \begin{tabular}{c|c|c|c|c}
                    &  SE   & ME    & MC    & SME \\ \hline 
    deterministic   & x     & x     &       &      \\
    dissipation     &       & x     & x     & x    \\ 
    backaction      &       &       &       & x    \\
    state size      & $n$   & $n^2$ & $n$   & $n^2$
    \end{tabular}
    \label{tab:my_label}
\end{margintable}

\subsection{Calibrations Protocols in Simulation}
This will be a perfect place, to show the possibilities between the different simulation schemes. If we evolve the same methods, we should see the difference. 

Some nice ones are:
\begin{itemize}
    \item T1 is good, the continuous one can be done in SME as well
    \item Spectroscopy Qubit + Resonator
    \item Photon Counting
    \item Efficiency 
\end{itemize}


\section{Check of the Dispersive approximation}

\subsection{The Punchout}

\section{Readout in Simulation}




\section{Readout Infidelity Budget}