\chapter{Conclusion}\label{chap:conclusion}
\textbf{The following is chatgpt of some bullets. Thats why is seems so prentious...}
In this thesis, we embarked on a comprehensive journey into the realm of superconducting readout, delving deep into the dispersive approximation and its underlying theory. Through the introduction of pivotal concepts such as the Lindblad equation for environmental interactions and the Stochastic Master Equation for trajectory analysis, we established a robust foundation for modeling this intricate process.

A pivotal aspect of our work was the calibration of a real-world qubit from the laboratory, enabling us to assign realistic parameters to our model. This not only bolsters the practicality of our findings but also establishes a direct link between theoretical insights and tangible experimental devices.

Upon conducting a thorough comparison with actual readout experiments, we observed a remarkable qualitative similarity in behavior. Further fine-tuning of parameters, tailored to the available computational resources, yielded closely aligned results. This convergence serves as a testament to the efficacy of our model in capturing the essence of real-world scenarios.

One of the paramount contributions of our study lies in the identification and quantification of factors contributing to SPAM errors. We discovered that in our specific system, temperature fluctuations at $\tau = 150 \text{ mK}$ accounted for approximately $22\%$ of the infidelity, while a characteristic decay time of $T_1 = 2.9 \text{ µs}$ contributed around $9\%$. Additionally, a reduced efficiency of $\eta = 13.5 \%$ led to a roughly $2\%$ impact on fidelity.

By conducting systematic adjustments, we demonstrated the potential of our approach in pinpointing avenues for enhancing fidelity. Notably, we discussed various strategies to mitigate and balance the influences of each error source. Our analysis suggests that, in the presence of fixed efficiency and decay rates, optimizing the readout duration stands as a pivotal step towards obtaining more representative estimates of potential improvements.

In essence, our work not only lays a robust theoretical foundation for superconducting readout but also offers practical insights into fine-tuning parameters for improved fidelity. As we navigate the frontier of quantum computing, these findings pave the way for more refined and efficient quantum information processing protocols.

% \chapter{Conclusion}
% \begin{itemize}
%     \item Covered the theory necessary to make a realistic model of superconducting readout in the dispersive approximation.
%     \item This includes the introduction of both the Lindblad equation for including interaction with the environment and the Stochastic Master Equation to get trajectories and model the back action on the qubit. 
%     \item A Qubit from the Laboratory has been calibrated such that the parameters for the model has realistic values. Furthermore, this also increases the applicability of the thesis since the conclusion are directly linked to a physical device. 
%     \item We have then compared the readout experiment finding that the qualitatively behaviour is very similar. By further tweaking the parameters slightly such that the experiment can be run on the available computational power, we also get similar results.
%     \item With the model, we have tried to run the experiment with and without some of the hypothesized contributors the SPAM errors. Here, we find that for the given system, the main contribution to spam error is Temperature of $\tau = 150 \text{ mK}$ giving around $22\%$ contribution to infidelity, characteristic decay time of  of $T_1 = 2.9 \text{ µs}$ giving around $9\%$ and a decreased efficiency of $\eta = 13.5 \%$ leading to contributions of around $2\%$.
%     \item By making marginal adjustments, we have shown that the method can help us identify the improvements to fidelity given improvements to the temperature, decay rate or inefficiency. We have also discussed different other methods to reduce and exchange the contributions from each of the sources. Here we argue that given efficiency and decay rate one should optimize the readout duration before altering the states to get a more representative estimate of the possible improvements. 
% \end{itemize}


\section{Next Steps}
During the learning, writing and coding the material and results for this thesis, my overview of the field and the method have developed significantly. Of course, this means that looking back at the thesis, there are many things I would have liked to do differently, in another order or maybe shelved a bit too early. This is just in addition, to all the projects and ideas that we did not even get to in this thesis. This section will therefore serve as an introduction to a few things which could serve as next steps for the project.

\subsection{Fitting the Model to Trajectories}
In this thesis, the model was made by calibrating the qubit using different methods for each parameter. By doing this, it is not a very convincing environment to try to load the parameters into the model and say that it qualitatively matches. Instead, we could have fitted the model to the given data. As seen in figure \ref{trajectories_and_qfunc}, by convolving the q-function with a Gaussian, we get a good estimate for distribution of the readout record at a given time. One could imagine using this to calculate the log.likelihood of each point at each time given a set of parameters. If we consider this a cost function, we can leverage the development of ODE integration techniques in deep learning libraries. This allow us to maximize the likelihood by tuning the parameters, where the gradients can be found by either autograd methods or by the adjoint state method. \cite{asdfasdfasedf}

New libraries for minimizing stochastic differential equation models have also been developed and are now already used to calibrate qubit parameters \cite{asadfasdfasdf}. Here the the set of equation from section \ref{sec:} was used to only run the simulation in a two dimensional Hilbert space. One could imagine expanding this to consider the resonator as well, but it will probably require large amonuts of computing power.


\subsection{Including Improved Strategies in Simulation} 
In chapter \ref{chap:readout_infidelity_budget} we discussed different strategies to swap good performing variables for others. It would definitely be beneficial to include these strategies in simulation. First, the amplitude and duration of the pulse should be optimized in the realistic setting before good estimates for improved devices can be extracted. The next steps will be to include $\ket{1}\to\ket{2}$ pulses as well as active reset. This would allow for a more flexible model that does not slack a few steps behind what is happening at the fridge. 

\subsection{Bigger Hilbert Space - High Power Simulations}
Lastly, we have been limited in our readout power, not by the quantum device in question, but by classical computation power used to run the simulation \footnote{A laptop, so it is probably not necessary to search far for something with a bit more power}. By having more computational resources (and time) it would be possible to also drive the resonator to a mean photon number of $\approx30$ or maybe even higher. This would give a better illustration of what is happening in the laboratory. Of course, we will need to check the dispersive approximation thoroughly when we start to add more photons.   