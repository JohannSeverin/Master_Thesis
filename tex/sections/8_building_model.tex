\chapter{Building a Model of the System}\label{chap:model}
With a theory of superconducting qubit readout and a set of parameters that describe our system with respect to this theory, we will in this chapter built up the simulation. We will will present this simulation work in three steps. First, we will take a look at four different methods of simulating the system. Next, we will argument for the use of the dispersive model compared to the full time-dependent Hamilton. Finally, we will adjust some parameters of the simulation such as the dimensions of our Hilbert Space and the size of the time steps.

\section{Different Simulation Approaches}
Throughout the first few chapter, we covered different ways of representing and numerically integrating a quantum system. In this chapter, we will summarize the few methods. Some of the properties are summed up in table \ref{tab:simulation_types}

\begin{itemize}
    \item \textbf{Unitary} - This is a time evolution of the Schrödinger Equation \ref{sec:scroedinger}. In Qutip this is done by the Adams algorithm which we covered in section \ref{sec:numerical_implementations}. This is the fastest and simplest to run, but does not support interaction with the environment.  
    \item \textbf{Lindblad Equation} - The Lindblad equation simulates the density matrix and allows us to include dissipation terms. Like the Schrödinger equation, the Lindblad Master Equation is also deterministic, so it is only necessary to run it once for each configuration. All dynamics can then be extracted from $\rho$. 
    \item \textbf{Monte Carlo} - The Monte Carlo method is described in sec \ref{sec:monte_carlo} and the main idea is that dissipation are applied stochastically and in between the dynamics are governed by the Schrödinger Equation. This allows for faster simulation than the Lindblad Master Equation. However, multiple trajectories will have to be taken to make sure the dynamics represent the full dynamics.
    \item \textbf{Stochastic Master Equation} - The Stochastic Master Equation is described in chapter \ref{chap:measurements} and the most complicated of the simulation tools. This includes the dynamics of the Lindblad Equation, but in addition also supports the weak measurement which is obtained during a readout.
\end{itemize}\todo{Citations}

\begin{margintable}[-10 cm]
    \centering
    \caption{Overview of what the different simulation schemes support. The abbreviations correspond to SE: Schödingers Equation, ME: Master Equation, MC: Monte Carlo, SME: Stochastic Master Equation}
    \vspace{0.3 cm}
    \begin{tabular}{c|c|c|c|c}
                    &  SE   & ME    & MC    & SME    \\ \hline 
    deterministic   & x     & x     &       &        \\
    dissipation     &       & x     & x     & x      \\ 
    mixed states    &       & x     &       & x      \\ 
    measurements    &       &       &       & x      \\ 
    state           & $\psi$& $\rho$&$\psi$ & $\rho$ \\
    state size      & $n$   & $n^2$ & $n$   & $n^2$
    \end{tabular}
    \label{tab:simulation_types}
\end{margintable}

\subsection{Comparing Simulations for $T_1$ Calibration}
\begin{figure*}[t]
    % \begin{minipage}{0.45\textwidth}
    \centering
    \includegraphics[angle=90, width=\linewidth]{Simulations/simulations_of_calibrations/Figs/qubit_T1.pdf}
    % \end{minipage}
    %     \begin{minipage}{0.45\textwidth}
    %     \centering
    %     \includegraphics[]{Simulations/simulations_of_calibrations/Figs/resonator_spectroscopy.pdf}
    % \end{minipage}
    \caption{Illustration of the $T_1$ and resonator spectroscopy run with different simulations schemes.}
    \label{fig:calibrations_in_simulation}
\end{figure*}
To perform a test, we have run the simulation for both the full Hamiltonian and the dispersive approximation. The result of these simulations can be seen in figure \ref{fig:calibrations_in_simulation} and the running time\footnote{On a laptop with processor Intel i7-1260P} can be seen in table \ref{tab:simulation_t1_running_time}. The simulations were run with Hilbert space size of ... \todo{This section still needs work}

\begin{margintable}
    \caption{Running time of the different simulation approaches to running the $T_1$ calibration scheme.}
    \vspace{0.3 cm}
    \centering
    \begin{tabular}{r|llll}
                    &  SE   & ME    & MC    & SME   \\ \hline 
    Full            & 1000  & 1000  & 1000  & 1000  \\
    Dispersive      & 1     & 1     & 1     & 1     \\ 
    \end{tabular}
    \label{tab:simulation_t1_running_time}
\end{margintable}
Deciding which simulation to use for which task is a trade-off between complexity and computation time. If one could ignore the interactions with the environment and have a completely unitary system, then there is no doubt that the Schrödinger equation should be used. When adding dissipation, we can either make use of the Lindblad Master Equation or the Monte Carlo solver. Here the benefit is a deterministic nature of Lindblad, where all information can be extracted from a single run. With a large Hilbert space this can however be tough, and we should instead consider the Monte Carlo solver.

The Stochastic Master Equation Solver was not used on the full system since it is a combination of the Lindblad and another stochastic element making it even more complex. With the dispersive approximation, it becomes manageable and only uses a few seconds to run. 

\todo{Some dynamics might become more clear, when we have the proper figures and times, so give this section a revamp}


% To compare the different simulation methods, we repeat some of the calibration protocols from chapter \ref{chapter:calibrations}. Since the goal of this thesis is to model the readout, we are focusing on replicating the features relevant for readout. Since the Stochastic Master Equation is expensive to run, we will only use it, when the measurement records are necessary. We will thus test the Schrödinger Equation, Monte Carlo and Lindblad equation on some simple calibration schemes. 

% The simulation are made by using the Hamiltonian for the full time-dependent Hamiltonian for the system given by equation \ref{eq:full_hamilton} and adding a contribution from a drive, where it is applicable. The parameters are taken from table \ref{tab:cali}made using $f_{01}$ ... from the cal

% A collection of the simulation results from resonator spectroscopy, $T_1$ calibration and ... are found on page \pageref{fig:big_figure_test}. We note the following :

% In calibrating photon decay rate $\kappa$ and the temperature, we use more data from a single trace than just its average expectation value. Thus we will also include simulation in the Stochastic Master Equation.  \todo{Use MC as well? Don't need it, but need an argument} 

% \todo{Comment on the figure and the results.}

% \begin{figure*}[h]
%     \centering
%     % \missingfigure{$T_1$ Calibration for the different simulations. }
%     % \label{fig:enter-label}
%     % \missingfigure{$T_2$ Calibration for the different simulations. }
%     \caption{Caption 2}
%     \label{fig:big_figure_test}
%     \includegraphics[]{Readout/Figs/Introduction.pdf}
%     \hspace{2 cm}
%     \includegraphics[]{Readout/Figs/Introduction.pdf}
%     \includegraphics[]{Readout/Figs/Introduction.pdf}
%     \includegraphics[]{Readout/Figs/Introduction.pdf}
% \end{figure*}



% \begin{figure}
%     \centering
%     \missingfigure{Calibration Scheme of $\kappa$ measurements}
%     \missingfigure{Calibration Scheme of $\tau$ IQ plot ? } 
%     \caption{Caption}
%     \label{fig:enter-label}
% \end{figure}

\FloatBarrier

\subsection{Q Function and Trajectories}
In section \ref{sec:IQ_phase_space}, we introduce the Q-Function to determine the phase space probability of finding the resonator with a specific $I, Q$ value set. It might be worth it to compare this Q function to the measurement record, we achieve by doing a stochastic master equation measurement.
\todo{This section needs a treatment}
\begin{figure*}[t]
    \centering
    \includegraphics[]{Simulations/readout_simulations/figures/qfunc_trajectories.pdf}
    \caption{Comparison of the Q Function and the scatter plot for a 10 ns readout record. In the top plot the Q Function distribution is shown at $t = 0, 200$ and $400 \text{ ns}$. In the two lower rows the the measurement record for 250 $\ket{0}$ and $\ket{1}$ trajectories are shown. Furthermore, the Q-Function is convolved by a 2d Gaussian with covariance matrix $2 \Delta t / \eta \identity$ to match the error of the records.}
    \label{fig:trajectories_and_qfunc}
\end{figure*}
\begin{itemize}
    \item A paragraph: Philosophical differences between measuring and not measuring. Lindblad is the average of many stochastic master equation. This is represented in the mixed state of the density matrix.
    \item We can covelute the Q function with a Gaussian, which we can find by comparing with the stochastic master equation. This can give the distribution of points at any time.
    \item The stochastic master equation is necessary for the inter-trajectory correlation. We could sample points from the Q-function, but it would as an example give us 90\% from the $0$ state and $10\%$ for the excited state. However, in reality one trajectory will have 100\% of its measurements from one distribution from just one of them.
\end{itemize}

\vspace{1 cm}
Old $\downarrow$ \\






When we consider the Lindblad equation, the result is a deterministic list of density matrices at each point in time. This is also a possiblity, when we integrate the Stochastic Master Equation, but in addition, we have the simulated trajectories which resemble the contious weak readout done in the laboratory. In order to compare these two methods, we will make use of the Q-function represented in \ref{sec:QFunc} which illustrates a 2-dimensional probability density of the quadratures in the resonator. 

If we simulate the readout process of the qubit-resonator system under the dispersive approximation, we can compare the measurement records from the Stochastic Master Equation with the evolution of the Q-function from the Lindblad Equation. This is summarized in figure \ref{fig:trajectories_and_qfunc}


The measurement record is given by a term proportional to the expectation value of the measured quantity and a noise term, which in the derivation from chapter \ref{sec:} was a Gaussian. For a perfect coherent state, this is closely to the distribution of its Q function. Especially in the hetereodyne measurements. For a state, we can get a somewhat approximation. We need to scale the standard deviation with eta. Furthermore, we get a reductioon depending on the time of the measurement. This gives the good idea of measuring.... \todo{This is really just me writing sentences which could be included}





\subsection{Validity of the Dispersive Approximation}
The full time-dependent Hamiltonian, we simulated to get the results shown in figure \ref{fig:calibrations_in_simulation} are expensive to run. The largest part of this process is to calculate the time-dependent function of the pulse. In the dispersive approximation which we covered in section \ref{sec:dispersive} we could eliminate the time-dependence of the pulse by going to another basis.

Before continuing with the approximation, we will first test whether it still exhibits the behaviours, we need to investigate the readout. 

\begin{figure}
    \centering
    \includegraphics[width = \textwidth]{Simulations/readout_simulations/figures/dispersive_approx.pdf}
    \caption{Caption}
    \label{fig:enter-label}
\end{figure}

\subsection{The Punchout}
In the dispersive limit, we consider that $g/(\omega_r - \omega_q) \ll 1$, which has be sufficient for us since we consider low-photon-number readout. If we were to increase this to the high power regime, one starts to see a lot of interesting physics. The point of increasing $n > n_{\text{crit}}$, the approximation is no longer good. Doing the experiment in the laboratory we see that the resonator at some point get completely decoupled from the qubit.

\begin{marginfigure}
    \centering
    \missingfigure{Punchout visualization}
    \caption{Caption}
    \label{fig:experiment_punchout}
\end{marginfigure}

In this regime the dispersive approximation is definetly wrong and we would have to resort back to the full time-dependent Hamiltonian. However, since the ciritcal photon number is $\approx 400$, we are nowhere near able to run these simulation anyways. 

\section{Size of the Hilbert Space}
The size of the Hilbert Space affects the complexity of the simulation significantly, since the entries of the density matrix scales as $n^2$ with $n$ the dimension of the Hilbert Space. However, picking the size is like most of these considerations a trade off between accuracy and speed of simulations. 

For the Qubit we need to have at a two dimensional system, but with temperatures around $100$ to $150 \text{ mK}$, there will also be non-negligible part in the second excited state of order $\approx 1\%$. Thus we will include this as well to have a three dimensional Hilbert Space for the Qubit.

For the resonator, we have to make sure, we do not miss any of the dynamics. Thus the size of the Hilbert space should be significantly larger than the maximum number of photon, we will have at any point during the simulation. The coherent states can be represented by a Poisson Distribution around the mean photon number \todo{Write Formula}, thus we take the maximum photon number and add around 5-10 dimensions more to make sure, we limit the numerical artifacts in the simulations.

\section{Readout in Simulation}
We are now in a position, where we have a qubit calibrated in experiment and recreated in simulation. By using the dispersive approximation, we can eliminate the time-dependence of the Hamiltonian to significantly speed-up the simulation process. And we have build up a framework of different simulation tools giving us a set off trade-offs between speed and complexity. \todo{This gives a bit conclusion-vibes}

Before using the simulation to learn about the system, we will first see if the readout fidelity in the two situations agree. Thus we set up a SME simulation giving us data corresponding to what we analyzed in chapter \ref{sec:improving_readout_fidelity}. One advantage of the simulation is that we already entered the rotating frame, so there is no need to demodulate the signal. Now we can repeat the process from earlier, and by using the matched weight filter, we can obtain the fidelity score.

\begin{figure*}
    \centering
    \includegraphics[]{Simulations/readout_simulations/figures/weigted_simulation.pdf}
    \caption{Caption}
    \label{fig:enter-label}
\end{figure*}

