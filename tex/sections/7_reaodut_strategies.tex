\chapter{Readout Strategies}
While considering the entire time-evolution dynamics can prove more insight in our system, it can also be used in different ways to optimize the system further. In this section, we will consider how to use the entire time-trace and knowledge of the dynamics to perform the best possible readout.

\section{Filtering and Weights}
\begin{itemize}
    \item Drawing borders, best will probably be to just use maximal seperation (Fischer)
    \item Filter comparison: how weighing the entire time trace can improve performance
    \item Going beyond linear models with Neural Networks. Will probably be more fun if we do multiplexing, but still... 
\end{itemize}


\section{Sacrificing Overhead for More Reliable Data}
This will be the place to cover, how we can use our overhead to get more reliable data. Since we have some T1 decay, that flips during the measurement. The edges will be very pure. Now we can sort the points by how certain we are. In the limit, this removes T1 errors in our calibration. Howevevr, state preparation errors will follow the other distributino and still be a source of errors. 

\section{Cloaking}
Up until now, the work done has been very passive: how do we use the information already available to learn more about the state of our qubit or the properties of our system in general. We will now change gears and start to think about, how one actively can change the dynamics of the qubit to improve the readout fidelity. In this section, we will consider the concept of cloaking a qubit in a cavity \cite{llado}. 

\subsection{Cloaked Qubit in Cavity}
\textbf{Need to write about qubit control in earlier section. Possibly sec. 3. } \\
Remembering that the qubit and resonator system has hamiltonian given by \ref{eq:HAMILTOIN}:
\begin{equation}
    H = \omega_r a^\dagger a + \sum_k \omega_k \ket{k}\bra{k} + g \hat{n}(a + a^\dagger)
\end{equation}
and we have two control lines, one for qubit and resonator respectively, which couples with the following interactions:
\begin{align}
    H_d^Q(t) &= \varepsilon_Q(t) \hat{n} \\ 
    H_d^R(t) &= \varepsilon_R(t) (a + a^\dagger) 
\end{align}
In regular dispersive readout as described in section \ref{sec: 5!} the resonator drive will take the form of a cosine pulse multiplied with an appropriate envelope function, while the qubit drive stands idly by. The idea behind cloaking is to use the qubit drive to cancel the interaction term $g\hat{n}(a + a^\dagger )$ of the hamiltonian effectively seperating the qubit and resonator. 


\subsection{Improved Readout using Cloaking}


\subsection{Simulations}

\subsection{Experiments}
