\chapter{Calibration Methods}
In order to use a qubit and in particular if we want to simulate said qubit, we need to learn the most important parameters. In this section, we will go through the different methods used for calibrating a qubit and the connected resonator in order to have a useful qubit along with the possibilities of simulating the system.


\section{Qubit Calibration}
Before using a qubit, it will be necessary to characterize it to the best of our ability, in this section, we will go over the standard methods for characterizing the necessary qubit and resonator properties. In this thesis experiments and simulations were done on a Transmon system.

\subsection{Spectroscopy}
In order to drive any qubit operation, the qubit frequency $f_{01}$ has to be determined with a high precision. In order to get the best estimate for the qubit frequency, we make a rabi style frequency, were we apply a pulse with a given frequency to the qubit. If we are in resonance, we should see the qubit moving between $\ket{0}$ and $\ket{1}$ and out of resonance no changes will be seen.

\begin{itemize}
    \item Qubit 
\end{itemize}

\subsection{Rabi}

Use this to find:
\begin{itemize}
    \item Proper pulses
    \item T1
\end{itemize}


\subsection{Decay Calibration}
How do we determine the calibration of T1. 

We get some measure from Rabi oscillations. But we follow this algorithm:

\begin{enumerate}
    \item Perform an x-gate
    \item Wait a time: $t$ 
    \item measure 
    \item repeat 1-3 $n$ times
    \item repeat 1-4 with increased $t$
\end{enumerate}

Fitting an exponential decay to the occupation in the excited state will give a measure 

\subsection{Dephasing Calibration}
There are two ways of calibrating the dephasing time. \textbf{Refer to the section in Lindblad}.

\paragraph{Ramsey Experiment}
This is a representative time of the qubit.

\paragraph{Echo Experiememt}
This is the limit if we allow activly chanigning the qubit to echo out the phases. We can thus eliminate low frequency noise.


\section{Resonator Calibration}
\subsection{Spectroscopy}
\begin{itemize}
    \item Single dip
    \item Double dip - dispersive shift
\end{itemize}

\subsection{Photon Counting}
Considering the dispersive hamiltonian of the qubit-resonator-system, Eq. \ref{eq:two_level_qubit_dispersive}, we have above primarily considered the shift of resonator frequency depending on the qubit state, but just moving the parenthesis to represent it in the form:
\begin{equation}
    H = \Tilde{\omega}_r a^\dagger a  + \left(\frac12 \Tilde{\omega}_{01} + \chi a^\dagger a\right)  \sigma_z
\end{equation}
making it apparent that while the qubit state moves the resonator, the opposite is also true. Since the qubit frequency moves with $\chi$ for each photon in the resonator, the photon number at a given amplitude can be calibrated.  

The protocol is carried out in the following way:
\begin{itemize}
    \item Drive the resonator at a given amplitude to fill it with photons.
    \item When the resonator is in its steady state apply an X-gate with a given frequency.
    \item Perform a regular readout
\end{itemize}
By sweeping over amplitude and X-gate frequency, one can find the resonance frequency at every amplitude. Since the resonance frequency is related to the qubit \\

\begin{marginfigure}[- 5 cm]
    \centering
    \includegraphics[width = 1.3 \textwidth]{tex/fig_for_text/section_6/photon_number_calibration.png}
    \caption{Simulation of photon number calibration protocol.  }
    \label{fig:photon_number_calibration}
\end{marginfigure}

\noindent
\textbf{We need the following Data}:
\begin{itemize}
    \item First calibrate the double dip and find the dispersive shift from above
    \item For the $\ket{0}$, we should drive the resonator at the resonator frequency and measure the $|I + iQ|$
    \item For the $\ket{1}$, we fill the resonator with photons and apply an x-gate with a given frequency. Scan over this frequency to find the resonance frequency as a function of amplitude $\to |I + iQ|$ measured above
    \item We should repeat this experiment where we apply another readout frequency to the most optimal (probably in between the two peaks). 
\end{itemize}

\section{Readout Efficiency}
The amplification chain from the qubit and resonator up to our digitization of the signal is by no means lossless. As mentioned in section \ref{sec: Amplifiers}\todo{Write section about amplifiers} there is a lower bound on the amount of noise added by an amplification. But in addition, the signal sees multiple amplification and thermalization at different stages ultimately giving us a much noisier signal than what is actually extracted from the qubit. To include this effect in our simulation and just to benchmark the amplification chain we will extract the quantum efficiency, $\eta$ by a method presented in \cite{bultink},\todo{Cite properly the Bultink et al. article} where the backaction of the qubit is compared to the Signal to Noise Ratio of the output signal. 

\begin{marginfigure}
    \centering
    \includegraphics{Figs/calibrations/efficiency/pulse_shape.png}
    \caption{The signal in the resonator during the readout signal.}
    \label{fig:efficiency_pulse_shape}
\end{marginfigure}

To use this method, one must first create a readout pulse with a signal that is zero at start and end: $S(t = 0) = S(t = T) = 0$. To limit decoherence during the readout, one would choose a short pulse and readout during a cooldown of the resonator for $\approx 5 / \kappa$.\todo{One can optimzie this further by applying a stimulated depopulation pulse. Write this in here if we do it.} The shape and resonator signal in pulse is shown in figure \ref{fig:efficiency_pulse_shape}. \todo{Add pulse shape as well}

Given the readout signal, the extraction now consist of two parts:
\begin{enumerate}
    \item Determine how the SNR changes when we increase the readout amplitude. With a signal starting and ending at 0, this should follow a linear relation: $\text{SNR}(\epsilon) = a \epsilon$.
    \item Investigate the backaction of the readout on the qubit. After a pulse, the coherence is related to the readout amplitude by the gaussian relation $|\rho_{01}(T, \epsilon)| = |\rho_{01}(T, 0)|e^{-\epsilon^2/2\sigma^2}$
\end{enumerate}

Which is related to the quantum efficiency by\cite{boltink}\todo{Derive either here, in appendix or above in the theory section}:
\begin{equation}
    \eta = \frac{a^2\sigma^2}{2}
\end{equation}

\subsection{Amplitude Dependence of SNR}
To determine the SNR from the readoutpulse, one does a simple experiment, which is close to what we have done before. \todo{Refer and make a connection} By initializing the qubit in $\ket{0}$ and then performing an $X$-gate to every other initialization before reading it out, one can see determine the separation between $\ket{0}$ and $\ket{1}$ (see figure \ref{fig:efficiency_SNR_experiment})

\begin{marginfigure}
    \centering
    \includegraphics{Figs/calibrations/efficiency/experiment_circuit_SNR.png}
    \caption{Circuit for experiment}
    \label{fig:efficiency_SNR_experiment}
\end{marginfigure}

Repeating the experiment $n$-times for different values of $\epsilon$, we obtain the results in figure \ref{fig:effiiency_results_SNR}, where the distribution for the last $\epsilon$ is shown together with the relation between calculated SNR and applied amplitude. Fitting a linear equation to the results, we obtain $a = ??$. \todo{Extract value from $a$ with errors and write samples, make prettier figure.} 

\begin{figure}
    \centering
    \includegraphics{Figs/calibrations/efficiency/SNR_result.png}
    \caption{Figure showing the experiment for extracting SNR as a function of readout ampltiude. An example at the highest amplitude is shown, where the two distributions are separated. The SNR is extracted as $\text{SNR}^2 = \expval{S_{\ket{0}} - S_{\ket{1}}}^2/\left(\expval{S^2_{\ket{0}}} +\expval{S^2_{\ket{1}}}\right)$. The other figure displays the result of this analysis for all values of $\epsilon$ as well as a linear fit applied to it.}
    \label{fig:effiiency_results_SNR}
\end{figure}



\subsection{Backaction from Readout Pulse}
\todo{When writing the Backaction section together with the stochastic equation, make sure everything needed here is covered. Connect this section back.}
For finding the relation between the amplitude and backaction, the qubit is initialized in $\ket{0}$ and a $R^x_{\pi/2}$ pulse is performed to send the qubit into $\frac{1}{\sqrt{2}}\left(\ket{0} + \ket{1}\right)$. With the pulse in this state, the density matrix will be:
\begin{equation}
    \rho(t=0) = \frac12 \begin{pmatrix}1 & 1 \\ 1 & 1\end{pmatrix}
\end{equation}
the readout drive will now be applied\footnote{Just as pulse because we're not interested in the signal}. During the readout signal the off-diagonal elements will be reduced by two factors: inherent dephasing by $T_2$-processes and qubit back action by the signal. To determine the amount of dephasing, we will perform a $\pi/2$ rotation around the axis $\phi$ in the x-y-plane before reading out the signal. The entire sequence is illustrated in figure \ref{fig:efficiency_dephasing_experiment}.

\begin{marginfigure}
    \centering
    \includegraphics{Figs/calibrations/efficiency/experiment_circuit_dephasing.png}
    \caption{Circuit for experiment}
    \label{fig:efficiency_dephasing_experiment}
\end{marginfigure}

Reading out the signal for different angles of $\phi$ will lead to a cosine shape: $\sigma_z = 2 |\rho_{01}(t= T)| \cos(\phi+\phi_0)$. And repeating for different readout amplitudes the dephasing due to backaction can be separated from the one due to $T_2$-dephasing which only has an effect of the amplitude. The results of the experiment can be seen in figure \ref{fig:efficiency_dephasing_experiment}, where the parameter from the important parameter for the Gaussian fit is the standard deviation: $\sigma = 0.0045 \pm ?$. \marginnote{Write the $\chi^2$ and $p-val$ for the fit in the marginnote.}

\begin{figure}
    \centering
    \includegraphics{Figs/calibrations/efficiency/dephasing_result.png}
    \caption{Results from the backaction experiment. The 2D scan is shown along with the cosine-way for amplitude $\epsilon = 0$ and $\epsilon = \epsilon_{max}$. Fitting the cosine function and plotting the amplitude as function of the readout amplitude gives the last figure, where they are are fitted by a gaussian distribution.}
    \label{fig:efficiency_dephasing_result}
\end{figure}

\todo{Redo figure.}

\subsection{Calculating the Efficiency}
Combining the results from the last two experiments, we find the readout efficiency to be:
\begin{equation}
    \eta = \frac{a^2\sigma^2}{2} = (1.5 \pm ?) \% 
\end{equation}

\subsection{Robustness Tests in Simulation ? }
\todo{Consider? }
\textit{Do this section make sense? Would be nice to check it, we can do it where we know the results in the efficiency and regulate the dephasing.}




\section{Calibration in Continuous Time}

\subsection{Transition Rates}
If we were to consider the dynamics over a longer time-interval, we would expect how to see the qubit change from $\ket{1} \leftrightarrow \ket{0}$ multiple times. If we were to consider, how often these transitions occur, we can determine the decay rate, and excitation rate allowing us to again extract $T_1$ and temperature $\tau$.

\textbf{Do analysis with data}



\subsection{In-measurement T1 }
Can we compare the decay-rate within a measurement to the decay outside. There is multiple reasons, this might change:
\begin{itemize}
    \item Resonator-Qubit interaction: the resonator and qubit are coupled, so when we are filling the resonator, we allow multiple form of decay/excitations through the qubit. This will lower the expect living time.
    \item Quantum Zeno Effect: Since the dynamics of the wave-function is linear but the stochastic nature of measuring is quadratic, a continuous measurement should somewhat prevent a transition. This will extent the expected lifetime of the qubit in both states with a factor related to the measurement efficiency $\eta$
\end{itemize}

\textbf{Do the experiment} where a continous measurement is made from 0 to 10 mikrosecond, followed up with another where a waiting windows is placed just before:
\begin{enumerate}
    \item Place the qubit in 0 or 1
    \item wait a number of mikroseconds
    \item do a contiouns measurement
\end{enumerate}
The ideas from before can now be used to fit the in-measurement decay rates, while we can use the points across measurements to control a more common decay rate which happens outside of measurements.