\chapter*{TO-DO list }
\begin{itemize}
    \item Write in stochastic master equation
    \item How can this be seen in homodyne/heterodyne measurements
    \item Go through the amplification channels and the minimum disturbance. + Vacuum state noise
    \item Write in results for the fits of the current data. Mostly overview, primaryly figures and comments
    \item Work on heterodyne simulations.
    \begin{itemize}
        \item Everlasting question - How do we sync this with data. mV -> intercavity photon field.
        \item different readout strategies, when T1 and efficiency is varied.
        \item create datasets for fit-testing
    \end{itemize}
    \item Start a PyTorch framework for stochastic equation fitting.
    \item Ask for continuous time measurements from the lab. 
\end{itemize}

\vspace{1 cm}
\noindent
\textbf{Thing to rewrite:}
\begin{itemize}
    \item  Chapter 2 and 3 should be combined to be a cQED section covering the dynamics and Hamiltonian of the system with proper approximations. Maybe we also want to add optics here? To cover for example the Husimi Q-function.
    \item Chapter 4 should be about open quantum system and should after the Lindblad Master equation cover the relevant dissipators for the Qubit System.
    \item Chapter 5 should start with weak measurements and go to the stochastic master equation. Then cover how the signal from the resonator is amplified through the feed line to ultimately give us a homodyne or heterodyne measurement. 
\end{itemize}