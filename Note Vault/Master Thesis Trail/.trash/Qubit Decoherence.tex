% Options for packages loaded elsewhere
\PassOptionsToPackage{unicode}{hyperref}
\PassOptionsToPackage{hyphens}{url}
%
\documentclass[
]{article}
\usepackage{amsmath,amssymb}
\usepackage{iftex}
\ifPDFTeX
  \usepackage[T1]{fontenc}
  \usepackage[utf8]{inputenc}
  \usepackage{textcomp} % provide euro and other symbols
\else % if luatex or xetex
  \usepackage{unicode-math} % this also loads fontspec
  \defaultfontfeatures{Scale=MatchLowercase}
  \defaultfontfeatures[\rmfamily]{Ligatures=TeX,Scale=1}
\fi
\usepackage{lmodern}
\ifPDFTeX\else
  % xetex/luatex font selection
\fi
% Use upquote if available, for straight quotes in verbatim environments
\IfFileExists{upquote.sty}{\usepackage{upquote}}{}
\IfFileExists{microtype.sty}{% use microtype if available
  \usepackage[]{microtype}
  \UseMicrotypeSet[protrusion]{basicmath} % disable protrusion for tt fonts
}{}
\makeatletter
\@ifundefined{KOMAClassName}{% if non-KOMA class
  \IfFileExists{parskip.sty}{%
    \usepackage{parskip}
  }{% else
    \setlength{\parindent}{0pt}
    \setlength{\parskip}{6pt plus 2pt minus 1pt}}
}{% if KOMA class
  \KOMAoptions{parskip=half}}
\makeatother
\usepackage{xcolor}
\setlength{\emergencystretch}{3em} % prevent overfull lines
\providecommand{\tightlist}{%
  \setlength{\itemsep}{0pt}\setlength{\parskip}{0pt}}
\setcounter{secnumdepth}{-\maxdimen} % remove section numbering
\ifLuaTeX
  \usepackage{selnolig}  % disable illegal ligatures
\fi
\IfFileExists{bookmark.sty}{\usepackage{bookmark}}{\usepackage{hyperref}}
\IfFileExists{xurl.sty}{\usepackage{xurl}}{} % add URL line breaks if available
\urlstyle{same}
\hypersetup{
  pdftitle={Qubit Decoherence},
  hidelinks,
  pdfcreator={LaTeX via pandoc}}

\title{Qubit Decoherence}
\author{}
\date{}

\begin{document}
\maketitle

\textbf{Primarily this is taken from A Quantum Engineers guide. It is
however modified to fit with the section where things are formulated in
terms of Lindblad Operators.}

If we consider a qubit in the picture of the Bloch Sphere, we have the
{} at the north pole and {} at the south pole. In this picture, we
consider two types of decoherence. One along longitudinal direction (the
{}-axis), and the transverse (in the {} - plane).

\hypertarget{longitudinal-relaxation}{%
\subsection{Longitudinal Relaxation}\label{longitudinal-relaxation}}

The longitudinal decay comes from transitions from {} by coupling to the
qubit in either the {} or in the {} channel. This could be from energy
exchange to and from the environment.

*If we were to sum over the coupling operators of the from creating or
annihilating a particle (or quasi particle) in the environment at the
cost of doing the opposite on the qubit.\\
{}If we were to trace out the environment now, we would be left with
Lindblad operators of the type {} or {} which are connecting to the
{}channels. Thus driving transitions relaxing the qubit from {} or
absorbing energy from the environment to go from {}.

However, since the state {} has a higher energy {} the excitation rate
{} is lower compared to the relaxation rate {} for low temperatures {},
The Boltzmann factor relating the two rates are given by: {} so for low
temperatures {} the steady state of the qubit will be the ground state
and the total longitudinal decay rate will be {}. Often this rate is
converted to a time as a measure of the stability of the qubit. Here it
is defined by {}.

\hypertarget{dephasing}{%
\subsection{Dephasing}\label{dephasing}}

If qubits instead connect longitudinally (along the {})

\end{document}
